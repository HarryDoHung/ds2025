\documentclass[a4paper,12pt]{article}

\usepackage{amsmath}
\usepackage{graphicx}
\usepackage{listings}
\usepackage{xcolor}
\usepackage{hyperref}
\usepackage{geometry}
\geometry{margin=1in}

\lstset{
    language=Python,
    frame=single,
    basicstyle=\ttfamily\footnotesize,
    keywordstyle=\color{blue},
    commentstyle=\color{gray},
    stringstyle=\color{green!50!black},
    numbers=left,
    numberstyle=\tiny,
    stepnumber=1,
    numbersep=5pt,
    showspaces=false,
    showstringspaces=false,
    showtabs=false,
    breaklines=true,
    breakatwhitespace=true,
    tabsize=4
}

\title{Practical Work 2}
\author{Do Cong Tuan Hung}
\date{\today}

\begin{document}

\maketitle

\section*{Goal}
The goal of this practical work is to upgrade the previously implemented TCP file transfer system to use Remote Procedure Call (RPC). The system now includes:
\begin{itemize}
    \item RPC server
    \item RPC client
    \item File transfer functionality over RPC
\end{itemize}

\section*{Protocol Design}
The RPC service facilitates file transfer using a client-server model. The workflow includes:
\begin{enumerate}
    \item The server registers an RPC service to handle file transfer requests.
    \item The client invokes the RPC method to request a file.
    \item The server sends the file to the client in chunks via the RPC service.
    \item The client writes the received chunks to a file.
\end{enumerate}

\section*{System Design}
The system is organized into the following components:
\begin{itemize}
    \item **Server**:
          \begin{itemize}
              \item Registers the RPC service.
              \item Handles file transfer requests from the client.
          \end{itemize}
    \item **Client**:
          \begin{itemize}
              \item Invokes the RPC method to request a file.
              \item Writes the received data to disk.
          \end{itemize}
\end{itemize}

\begin{figure}[h!]
    \centering
    \includegraphics[width=\textwidth]{system_organization.png}
    \caption{System Organization}
    \label{fig:system_organization}
\end{figure}

\section*{Implementation}
The implementation is done using the Python `xmlrpc` library. Below are the code snippets for the server and client:

\subsection*{Server Code}
\begin{lstlisting}
from xmlrpc.server import SimpleXMLRPCServer

HOST = '127.0.0.1'
PORT = 1234
BUFFER_SIZE = 1024
FILE_TO_SEND = 'testing_file.txt'

def send_file():
    with open(FILE_TO_SEND, 'rb') as file:
        while (data := file.read(BUFFER_SIZE)):
            yield data

server = SimpleXMLRPCServer((HOST, PORT))
print(f"RPC Server running on {HOST}:{PORT}")

server.register_function(send_file, "send_file")
server.serve_forever()
\end{lstlisting}

\subsection*{Client Code}
\begin{lstlisting}
import xmlrpc.client

HOST = '127.0.0.1'
PORT = 1234
OUTPUT_FILE = 'received_file.txt'

server = xmlrpc.client.ServerProxy(f"http://{HOST}:{PORT}")
print(f"Connected to RPC server at {HOST}:{PORT}")

with open(OUTPUT_FILE, 'wb') as file:
    for chunk in server.send_file():
        file.write(chunk)

print("File received and saved successfully.")
\end{lstlisting}

\section*{Roles and Responsibilities}
\begin{itemize}
    \item **Server**: Provides the `send_file` RPC method, which streams the file in chunks to the client.
    \item **Client**: Invokes the `send_file` method to receive the file and writes it to disk.
\end{itemize}

\section*{Testing}
\subsection*{Local Testing}
The RPC server and client scripts were run on the same machine. The following steps were performed:
\begin{enumerate}
    \item The server was started, displaying `RPC Server running on 127.0.0.1:1234`.
    \item The client connected to the server and successfully received the file.
    \item The received file matched the original file.
\end{enumerate}

\subsection*{Network Testing}
The server and client were run on different machines connected to the same network. The client successfully received the file via the RPC service.

\section*{Conclusion}
The RPC file transfer system was successfully implemented and tested. The system provides a robust file transfer mechanism using Python's `xmlrpc` library.

\section*{References}
\begin{itemize}
    \item Python XML-RPC Documentation: \url{https://docs.python.org/3/library/xmlrpc.html}
    \item LaTeX Documentation: \url{https://www.latex-project.org/}
\end{itemize}

\end{document}
