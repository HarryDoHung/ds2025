% !TEX root = 01_tcp_file_transfer.tex

\documentclass[a4paper,12pt]{article}

\usepackage{amsmath}
\usepackage{graphicx}
\usepackage{listings}
\usepackage{xcolor}
\usepackage{hyperref}
\usepackage{geometry}
\geometry{margin=1in}

\lstset{
    language=Python,
    frame=single,
    basicstyle=\ttfamily\footnotesize,
    keywordstyle=\color{blue},
    commentstyle=\color{gray},
    stringstyle=\color{green!50!black},
    numbers=left,
    numberstyle=\tiny,
    stepnumber=1,
    numbersep=5pt,
    showspaces=false,
    showstringspaces=false,
    showtabs=false,
    breaklines=true,
    breakatwhitespace=true,
    tabsize=4
}

\title{Practical Work 1}
\author{Do Cong Tuan Hung}

\begin{document}

\maketitle

\section*{Goal}
The goal of this practical work is to implement a 1-to-1 file transfer system over TCP/IP using socket programming in Python. The system includes:
\begin{itemize}
    \item One server
    \item One client
    \item File transfer functionality
\end{itemize}

\section*{Protocol Design}
The system uses a client-server model. Below is the flow of interaction:
\begin{enumerate}
    \item The server starts and listens on a specific IP address and port.
    \item The client connects to the server and requests the file.
    \item The server sends the file in chunks to the client.
    \item The client writes the received chunks to a file.
\end{enumerate}

\section*{System Design}
The client and server are implemented as follows:
\begin{itemize}
    \item **Server**:
          \begin{itemize}
              \item Creates a socket, binds it to a port, and listens for client connections.
              \item Sends the file in chunks to the connected client.
          \end{itemize}
    \item **Client**:
          \begin{itemize}
              \item Connects to the server using its IP and port.
              \item Receives the file and writes it to disk.
          \end{itemize}
\end{itemize}

The diagram below shows the client-server interaction:

\begin{figure}[h!]
    \centering
    \includegraphics[width=\textwidth]{diagram.png}
    \caption{Client-Server Protocol Workflow}
    \label{fig:protocol}
\end{figure}

\section*{Implementation}
The implementation is done in Python using socket programming. Below are the code snippets for the server and client:

\subsection*{Server Code}
\begin{lstlisting}
import socket

HOST = '127.0.0.1'
PORT = 1234
BUFFER_SIZE = 1024
FILE_TO_SEND = 'testing_file.txt'

server_socket = socket.socket(socket.AF_INET, socket.SOCK_STREAM)
server_socket.bind((HOST, PORT))
server_socket.listen(1)
print(f"Server listening on {HOST}:{PORT}")

conn, addr = server_socket.accept()
print(f"Connection established with {addr}")

with open(FILE_TO_SEND, 'rb') as file:
    while (data := file.read(BUFFER_SIZE)):
        conn.sendall(data)

print("File sent successfully.")
conn.close()
server_socket.close()
\end{lstlisting}

\subsection*{Client Code}
\begin{lstlisting}
import socket

HOST = '127.0.0.1'
PORT = 1234
BUFFER_SIZE = 1024
OUTPUT_FILE = 'received_file.txt'

client_socket = socket.socket(socket.AF_INET, socket.SOCK_STREAM)
client_socket.connect((HOST, PORT))
print(f"Connected to server at {HOST}:{PORT}")

with open(OUTPUT_FILE, 'wb') as file:
    while True:
        data = client_socket.recv(BUFFER_SIZE)
        if not data:
            break
        file.write(data)

print("File received and saved successfully.")
client_socket.close()
\end{lstlisting}

\section*{Testing}
\subsection*{Local Testing}
Both server and client scripts were run on the same machine. The following steps were performed:
\begin{enumerate}
    \item The server was started, and it displayed `Server listening on 127.0.0.1:8080`.
    \item The client connected to the server and successfully received the file.
    \item The received file matched the original file.
\end{enumerate}

\subsection*{Network Testing}
The server and client were run on different machines connected to the same network. The client successfully received the file over the network.

\section*{Conclusion}
The TCP file transfer system was successfully implemented and tested. The system ensures reliable file transfer between a server and a client using Python socket programming.

\section*{References}
\begin{itemize}
    \item Python Socket Documentation: \url{https://docs.python.org/3/library/socket.html}
    \item LaTeX Documentation: \url{https://www.latex-project.org/}
\end{itemize}

\end{document}
